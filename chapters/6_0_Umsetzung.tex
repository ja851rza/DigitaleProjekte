\chapter{Handlungsempfehlung}
\label{chap:handlungs}

	Wie in der Einleitung bereits angedeutet beschäftigt sich die gicom AG mit der Implementierung von Standardsoftware, sowie deren Customizing. Da es sich bei dem von der gicom vertriebenen Produkt um ein im gewissen Rahmen erweiterbares Produkt handelt, kann es zu unterschieden in den Aufgabenstellungen kommen. Jedoch sind diese Abweichungen überschaubar. Bei der Implementierung buchen die meisten Kunden alle Module, da diese sich ergänzen. Nur wer alle Module besitzt kann das volle Potential der Software ausschöpfen. 
	Aus dem oben genannten Gründen kann das Customizing auch in drei Teile unterteilt werden. 
	
	\begin{itemize}
		\item Customizing \ac{AD}
		\item Customizing \ac{RTM}
		\item Customizing \ac{ANW}
	\end{itemize}

	Jedes dieser Tools sollte über einen eigenen \acs{PSP} verfügen. Diese klare Abgrenzung hilft der gicom nicht nur eine erhöhte Übersichtlichkeit zu bewahren, sondern verschafft auch Klarheit in Bezug auf die Zugehörigkeit der Aufgaben. Da das Modul der \ac{AD} die Voraussetzung schafft für die anderen Module sollte dies zuerst angegangen werden. Genau wie es im \autoref{chap:umsetzung} beschrieben wurde sollte hier ein phasenorientierter Ansatz verfolgt werden. Daher auch die unterschiedlichen Phasen als Grundlage für die weitere Aufgabenfindung gelten. Die Findung der unterschiedlichen \ac{AP} ist ein weiterer wichtiger Bestandteil des Prozesses. Neben den zur Verfügung gestellten Grundpfeilern des Customizings verfügt die gicom über ein immenses Set an Erfahrung und Aufgaben im Intranet, die aus vergangenen Projekten gesammelt wurden. Die Erläuterung beziehungsweise die Erstellung eines \acs{PSP} auf dieser Grundlage würde den Rahmen der Arbeit überschreiten. Indem der Autor eine Vorgehensweise für die korrekte Definition der \acs{AP} zur Verfügung gestellt hat, soll sichergestellt werden, dass nichts in bei der Erhebung vergessen wird. 
	Die \acs{PSP} der Tools \acs{RTM}, sowie \acs{ANW} sollten nach dem gleichen Verfahren erstellt werden. 
	