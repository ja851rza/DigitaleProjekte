\chapter{Fazit}

	Ziel dieser Arbeit war es herauszufinden, wie ein Projekt aufgebaut ist um darauf basierend ein generisches Konzept zu erstellen, welches für das Customizing eingesetzt werden kann. 
	
	Dazu wurden im \autoref{chap:projektablauf} die Zusammensetzung eines Projektes beschrieben. Hierbei wurde unter anderem auf die einzelnen Phasen eingegangen, auch der \ac{PSP} sowie das \ac{AP} wurden erläutert. Das \autoref{chap:SpecCustom}  hat die Themenbereiche des Customizings beleuchtet. Basierend auf diesen beiden Kapiteln konnten dann Erkenntnisse für die Umsetzung einer Strategie im \autoref{chap:umsetzung} gezogen werden. Das entwickelte Modell konnte dann im \autoref{chap:handlungs} in den Kontext der gicom gebracht werden.  
	
	Abschließend lässt sich festhalten, dass das Ziel dieser Arbeit, ein generisches Modell zu erstellen, teilweise erfüllt werden konnte. Es konnte ein Grundgerüst eines \ac{PSP} für das Customizing erstellt werden, jedoch hat der Rahmen dieser Arbeit es nicht zugelassen, sich tiefer mit den detaillierten Aufgabenstellungen des Customizings zu beschäftigen. Daher konnte auch kein Detaillierterer \acs{PSP} erstellt werden.  