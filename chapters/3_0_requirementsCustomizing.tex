\chapter{Spezifika von Customizing Projekten}
\label{chap:SpecCustom}

	Dieses Kapitel befasst sich mit der Fragestellung \dritteUnterfrage Hier soll ein Einblick in die Aufgaben eines Consultants gegeben werden, sowie herausgearbeitet werden, welche Aufgaben zu erledigen sind.  
	
\section{Customizing}

	Das Customizing befasst sich laut dem Gabler Wirtschaftslexikon mit der \enquote{Anpassung von Standardsoftware an kundenindividuelle Anforderungen.}\autocite{Lackes_customizing}
	Das Ziel ist es, eine programmierte Software mit kundenindividuellen Daten zu befüllen. Da es sich bei der gelieferten Standardsoftware, im Gegensatz zu einer Individualsoftware, um eine nicht an die Kundenumgebung angepasste Software, welche an unterschiedliche Kunden angepasst werden kann, handelt, muss eine Anpassung vorgenommen werden. 
	Das Customizing einer Software verlangt auf der einen Seite die Anpassung der Software selbst. Indem unterschiedliche Module ausgewählt werden können, kann ein Kunde die von ihm benötigten auswählen, wodurch nur die in Rechnung gestellt werden, die wirklich genutzt werden. Auf der anderen Seite müssen die Daten in der Software angepasst werden. Hierbei handelt es sich um Schnittstellen, Spracheinstellungen, sowie \ac{UI} Einstellungen. Abschließend müssen Daten an das an den Kunden angepasste System eingespielt werden. Dies passiert entweder durch die Nutzung des neuen Systems oder durch Migration von Daten aus dem alten auf das neue System. \autocite[668-669]{Hansen2009} 
	
\section{SAP Customizing}
\label{sec:sapCustomizing}
	
	Für eine ordnungsgemäße Nutzung der SAP Software muss stets ein Customizing vorgenommen werden. Im Rahmen der Transformation der von den R/3 auf die S/4 HANA Systeme hat die SAP einen eigenen Ansatz des Customizing herausgebracht. Der unter dem Namen SAP Activate laufende Ansatz umfasst sechs Phasen, sowie einen Iterations-Kreis der Phasen unabhängig ist. Die benennung fand eir folgt statt: Discover, Prepare, Explore, Realize, Deploy, sowie Run\autocite[331]{Denecken2020}

\section{Aufgaben im Customizing}
	\label{sec:aufgabenCustomizing}
	
	Für derartige Anpassungen der Standardsoftware fallen die folgenden Arten von Aufgaben an: 
	
	\begin{itemize}
		\item Modulauswahl
			\subitem Auswählen ist eine zentrale Aufgabe im Customizing, da nicht immer klar ist welche Module der Kunde benötigt. Daher müssen anhand von Workshops mit unterschiedlichen Abteilungen Workshops durchgeführt werden.
			
		\item Setzen von Parametern
			\subitem Nachdem bekannt ist, welche Module ein Kunde benötigt, werden diese Module implementiert und mit grundlegenden Parametern zum lauffähig gemacht. 
			
		\item Grundeinstellungen
			\subitem Diese Einstellungen beziehen sich auf Länderspezifika.
			
		\item Abbildung unternehmensspezifischer Datenstrukturen im System
			\subitem Jedes Unternehmen verfügt über eigene Strukturen und Abläufe. Diese Aufgaben ziehen sich von der Erstellung einer Rechnung über die Abrechnung eines Artikels bis hin zu der Anspruchsermittlung für eine bestimmte Periode. Dieser Aufgabenbereich ist der Umfangreichste.
		
		\item Abbildung unternehmensspezifischer Prozesse im System
			\subitem Der Kunde besitzt häufig andere Prozesse, welche nicht den neuen Standards entsprechen, um eine Effizienzsteigerung garantieren zu können müssen in einigen Fällen auch die Internen Prozesse Angepasst werden. 
		
		\item Schulung der Mitarbeiter
			\subitem Die Schulung ist auch ein äußerst signifikanter Teil der Implementierung einer neuen Software, da der Erfolg der Software von der Akzeptanz der Mitarbeiter abhängt. 
	\end{itemize}