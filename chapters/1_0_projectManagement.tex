\chapter{Funktionsweise des Projektmanagements}


	In diesem Kapitel wird geklärt welche Merkmale Projektmanagement besitzt. Hierfür wird zunächst der Begriff  definiert, um im Anschluss das digitale Projektmanagement davon abzugrenzen. Es werden außerdem unterschiedliche agile Methoden aufgezeigt die es auf dem Markt gibt und Rückschlüsse gezogen. 

\section{Projekt Management}
\label{sec:projectMngmt}

	Um das Projektmanagement besser verstehen zu können, muss erst einmal der Begriff Projekt genauer erläutert werden. Ein Projekt wird in der DIN ISO Norm als 
	
	
	


\section{Digitales Projektmanagement}

Auch im digitalen Projektmanagement entsprechen Projekte den von Madauss genannten Kriterien, jedoch finden diese Projekte ausschließlich im digitalen Kontext statt. Dabei ist es nicht von belangen, ob es um digitale Produkte (Musik, Spiele, Plattformen) geht, um digitale Dienstleistungen (Self Service, Information durch Chatbot) oder um eine Kombination aus beidem. Das Ziel eines digitalen, sowie eines normalen Projektes ist die Wertsteigerung. 
\cite[S.33]{madauss2019}

\todo{Methoden des Projektmanagements agile Methoden}
\todo{Es muss auch eine Planungsmethode wie Plan... gezeigt werden}

\section{Projektplanung}

	