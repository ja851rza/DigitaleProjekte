\thispagestyle{plain}
\chapter{Abstract}
\label{sec:einleitung}

	
	Der Projekterfolg hängt heutzutage maßgeblich von der Termintreue , der Budgettreue sowie dem Endergebnis ab. Wird einer dieser Faktoren vernachlässigt, so kann ein Projekt sich verzögern oder sogar abgebrochen werden. Dies hat nicht nur eine negative Auswirkung auf das Unternehmen, welches das Projekt durchgeführt hat, sondern verschlingt auch Unmengen an Ressourcen die anderweitig eingesetzt werden können. Um dieses Geschehnis vorzubeugen, entstanden in den vergangenen Dekaden einige Ansätze des Projektmanagements. Jeder Ansatz verfügt über seine eigenen Werkzeuge die einem Projektmanager dabei helfen ein Projekt zum richtigen Zeitpunkt, mit dem richtigen Budget, sowie der richtigen Qualität abzuschließen. Im Bereich des Custoimzings, in dem Kunden sich darauf verlassen, das die Anpassung so effizient wie möglich erstellt wird, sodass mit dem System fehlerfrei gearbeitet werden kann, ist eine genaue Terminierung, sowie eine damit einhergehende Transparenz wichtig. 

	Die gicom AD ist ein SAP Beratungshaus mit Hauptsitz in Overath. Die mittlerweile mehr als 70 Beschäftigten beraten Kunden rund um das Thema Verhandlungsplanung, Verhandlungsvorbereitung, Verhandlungsstrategieentwicklung, sowie der Abrechnung. Bei der von der gicom zur Verfügung gestellten Software handelt es sich um eine Standardsoftware, welche zusammen mit dem Kunden angepasst wird. Die Software teilt sich in drei unterschiedliche Module von denen sich eines mit der Digitalisierung, das zweite mit der Berechnung unterschiedlicher Ansprüche, sowie das dritte mit der Verhandlung selbst beschäftigt. Jedes dieser Module benötigt eine separate Anpassung an die Unternehmensumgebung des Kunden. Ziel dieser Arbeit ist es ein Strukturiertes vorgehen im Customizing zu erkennen, anhand dessen ein generisches Modell entstehen kann welches dem Benutzer eine Handlungsempfehlung an die Hand gibt. Dadurch soll die Aufgaben Findung  erleichtert sowie eine Stringenz bei der Notation geschaffen werden.
	
	Die Wissenschaftliche Frage, die sich aus dem Titel der Arbeit ableitet, lautet:
	
	\begin{quote}
		\wissenschaftlicheFrage
	\end{quote}

	Für eine strukturierte Bearbeitung der Wissenschaftlichen Frage wurden drei Unterfragen gebildet, anhand derer die Ausarbeitung gegliedert wird. 
	
	\begin{enumerate}
		\item \ersteUnterfrage
		\item \zweireUnterfrage
		\item \dritteUnterfrage
	\end{enumerate}

	Die Fragen werden der Reihenfolge entsprechend abgearbeitet. Im \autoref{chap:projektablauf} wird somit geklärt, wie ein Projekt aussieht. Das \autoref{chap:SpecCustom} befasst sich mit den einzelnen Phasen eines Projektes. Hier werden die verschiedenen Ansätze des Projektmanagement erläutert. Während im \autoref{chap:umsetzung} ein Modell, zur allgemeinen Lösung eines solchen Problems, erstellt wird, befasst sich das \autoref{chap:handlungs} mit der Anwendung dieses Models auf die gicom AG. 
	

	
